\documentclass[12pt,-letter paper]{article}                       \usepackage{gvv}
\begin{document}
\begin{enumerate}
	\item Let $n$ be a positive integer $a_1,\dots,a_k\brak{k\geq 2}$ be distinct integer in the set $\cbrak{1,\dots, n}$ such that $n$ divides $a_i \brak{a_{i+1}-1}$ for $i=1,\dots, k-1$. Prove that $n$ does not divide $ a_k \brak {a_i-1}$.
\item Let $ABC$ be a triangle with circumcentre $O$. The points $P$ and $Q$ are interior points of the sides $CA and AB$, respectively. Let $K,L$ and $M$ be the midpoints of the segments $BP$, $CQ$ and $PQ$, respectively, and let $\lceil$ bethe circle passing through $K,L$ and $M$. Suppose that the line $PQ$ is tangent to the circle $\lceil$. Prove that $OP=OQ$.
\item Suppose that $s_1, s_2, s_3,\dots $ is a strictly increasing sequence of positive integers such that the subsequences \\
	$s_{s1}, s_{s2}, s_{s3},\dots$ and $s_{s1+1}, s_{s2+1}, s_{s3+1},\dots$
	   \\
are both arithmetic progressions. Prove that the sequence $s_1,s_2,s_3$,...is itself an arithmetic progession.
\item Let $ABC$ be a triangle with $AB=AC$. The angle bisectors of $\angle CAB$ and $\angle ABC$ meet the sides $BC$ and $CA$ at $D$ and $E$, respectively. Let $K$ be the incentre of triangle $ADC$.Suppose that $\angle BEK$=$45\degree$. Find all possible values of $\angle CAB$.
\item  Determine all functions $f$ from the set of positive integers to the set of positive integers
such that, for all positive integers $a$ and $b$, there exists a non-degenerate triangle with sides of lengths
		\\$a, f (b)$ and $f (b+f(a)-1).$ \\
		$(A triangle is non-degenerate if its vertices are not collinear)$.
\item Let $a_{1}$, $a_{2}$,\dots, $a_{n}$ be distinct positive integers and let $M$ be a set of $n-1$ positive integers not containing $s = a_{1}+a_{2}$+ \dots + $a_{n}$. A grasshopper is to jump along the real axis, starting at the point 0and making $n$ jumps to the right with lengths $a_1, a_2\dots,a_n$ in some order. Prove that the order can be chosen in such a way that the grasshopper never lands on any point in $M$.
\end{enumerate}
\end{document}



